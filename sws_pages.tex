\documentclass[12pt, a5paper, parskip=half-, footheight=1.4cm]{scrartcl}

\usepackage{scrlayer-scrpage} % Manage headers and footers in Koma-Script document classes

\usepackage[bmargin=2.5cm, lmargin=1.7cm, rmargin=1.7cm, tmargin=2cm, footskip=2cm]{geometry} % Set margins and footer placement

\usepackage{tikz}
\usetikzlibrary{arrows.meta} % Customize arrowheads in in Write the Outline example
\usetikzlibrary{shapes} % Draw the node in the fancy \pagemark

\usepackage[skins]{tcolorbox} % Add background image to tikz node in color cover

\usepackage{fontspec} % Use system fonts.  Not compatible with pdflatex. Use XeLaTeX instead!

\usepackage{enumitem} % Adjust formatting of description environment items
\usepackage{multicol} % Format list of playtesters in two columns

\usepackage[type={CC}, version={4.0}, modifier={by}]{doclicense} % Add text and icons for creative commons license

\usepackage{eso-pic}

\usepackage[hidelinks]{hyperref} % Add hyperlinks to the pdf file. This should usually be the last package loaded before \begin{document}
\usepackage[xspace]{ellipsis} % Properly typeset ellipses. This package is one of the few packages that must be loaded after hyperref


% Define command to set value of a variable that contains the version number
% This is meant to mimic the syntax used for the built-in \title{} and \author{} commands.
\makeatletter
\newcommand{\version}[1]{\newcommand{\@version}{#1}}
\makeatother

%Define a fancy \pagemark with some decoration around the page number.
%\renewcommand{\pagemark}{
%	\begin{tikzpicture}
%		\setmainfont[Scale=1.5]{Bilbo Swash Caps}
%		\foreach\i/\j in {0/90, 72/162, 144/234, 216/306, 288/18}
%			\node[inner sep=0, rotate=\i] at (\j:2.6ex) {W};
%		\setmainfont{Tex Gyre Heros}
%		\node (b) at (0,0) {\arabic{page}};
%	\end{tikzpicture}
%}

% Set header and footer text
\clearpairofpagestyles
\makeatletter
\chead*{\normalshape \footnotesize \footnotesize Version \@version}
\makeatother
\cfoot{\normalshape \pagemark}

% Adjust spacing before and after section headings
\RedeclareSectionCommand[
  runin=false,
  beforeskip=0.5\baselineskip,
  afterskip=0.25\baselineskip
]{section}

% Adjust spacing before and after subsection headings
\RedeclareSectionCommand[
  runin=false,
  beforeskip=0.5\baselineskip,
  afterskip=0.25\baselineskip
]{subsection}

% Adjust spacing before and after subsubsection headings
\RedeclareSectionCommand[
  runin=false,
  beforeskip=0.5\baselineskip,
  afterskip=0.25\baselineskip
]{subsubsection}

% Adjust formatting of description environment
\setlist[description]{font=\normalfont\bfseries, leftmargin=0pt}

% Adjust formatting of itemize environment
%itemize environments should be followed immediately by a \vspace{1ex} command.
\setlist[itemize]{noitemsep,nolistsep,  leftmargin=0.68cm,topsep=-1ex}

\setkomafont{section}{\setmainfont{URW Classico}\LARGE\bfseries}
\setkomafont{subsection}{\setmainfont{URW Classico}\Large\bfseries}
\setkomafont{subsubsection}{\setmainfont{URW Classico}\large\bfseries}

\definecolor{paper}{HTML}{F1EBDB}

% Set some environment variables for use in the title page
%\title{\phantom{g}SOCRATIC\phantom{g}\\[0.45ex]WORLDBUILDING\\[0.45ex]SYSTEM}
\title{\phantom{g}Socratic\phantom{g}\\[0.075ex]Worldbuilding\\[0.075ex]System}
%\subtitle{A tabletop roleplaying game \\ \smallskip about fate and destiny}
\author{Michael Purcell}
\version{0.1}

\begin{document}

% Colour Cover
%\begin{titlepage}
%		\makeatletter
%		\thispagestyle{plain}
%		\enlargethispage{3.5\baselineskip} % Move the bottom line (author and date) down a bit
%
%	\AddToShipoutPictureBG*{
%		\begin{tikzpicture}[remember picture, overlay]
%			\draw[fill stretch image=Images/black_marble_texture_vertical.jpg]  (current page.south west) rectangle (current page.north east);
%		\end{tikzpicture}
%	}
%
%%		\pagecolor{black}		
%		
%         \setmainfont{Cinzel Decorative}
%	    \centering{
%			{\fontsize{60}{72}\selectfont \textcolor{violet!75}{\@title}}
%		}
%
%		\setmainfont{URWClassico}
%		\vspace{5mm}
%		\centering{\Large{{\textcolor{lightgray}{\@subtitle}}}}
%
%
%		\vfill
%
%		\includegraphics[scale=3.81]{Images/comet_diagram_violet.pdf}
%
%		\vfill
%		
%		\Large{{\textcolor{lightgray}{\@author}}}
%		\makeatother
%\end{titlepage}

% BW Cover
\begin{titlepage}
	\makeatletter
	\thispagestyle{plain}
		\enlargethispage{2.5\baselineskip} % Move the bottom line (author and date) down a bit
		
         \setmainfont[Scale=4]{URW Classico-Bold}
	    \centering{\@title}

		\vfill

		\includegraphics[scale=1.1]{sws_logo.pdf}

		\vfill
		
		\setmainfont{URW Classico}\Large{{\textcolor{black}{Designed by \@author}}}
		\makeatother
\end{titlepage}

\setmainfont{URW Classico}
\normalsize

%\AddToShipoutPictureBG{
%	\begin{tikzpicture}[remember picture, overlay]
%		\draw[fill stretch image=Images/paper_texture_vertical.jpg]  (current page.south west) rectangle (current page.north east);
%	\end{tikzpicture}
%}

%\pagecolor{paper}

%\RedeclareSectionCommand[
%  runin=false,
%  beforeskip=0.5\baselineskip,
%  afterskip=-0.25\baselineskip
%]{section}

\center

\setcounter{page}{1}

\raggedright
\section*{Introduction} \label{section:introduction}
The Socratic Worldbuilding System (SWS) is a tool for creating one-page tabletop roleplaying games.

Every Socratic Worldbuilding Game (SWG) is about disruptive change.
In an SWG, the players portray characters who have gathered to discuss the ways in which they have been affected by such a change.
By answering a series of questions about the nature of the disruptive change, they will describe the world in which the game is set.

All SWGs comprise the same component parts: a setting, a disruptive change to that setting, characters, and topics of discussion.
In the following sections, we will use four SWGs as running examples.
The example games are: Staff Meeting, Parliament of Dragons, Kill the Beast, Not From Around Here.
Complete versions of all four example games can be found in the appendices to this rulebook.

%\subsection*{Materials} \label{subsection:materials}
%Socratic Worldbuilding Games generally require few, if any, material components to play. The Socratic Worldbuilding System is similarly lightweight. All that is required is a few index cards and something to write with. 

\newpage

\subsection*{Setting}
A setting is a description of the world in which the events of a story will take place.
In an SWG, the setting should provide a rough sketch of the world that the characters inhabit.
The players will fill in the details of the game's setting as they play.

\subsubsection*{Examples of Settings}
\begin{description}
\item[Staff Meeting:] A fantastical school of witchcraft and wizardry.
\item[Parliament of Dragons:] An island ruled by territorial dragons.
\item[Kill the Beast:] A village on the edge of a mysterious forest.
\item[Not From Around Here:] A luxury resort on a tropical island.
\end{description}

\newpage

\subsection*{Disruptive Change}
A disruptive change is something that changes some aspect of the setting.
This change is something that will significantly affect the players' characters.

\subsubsection*{Examples of Disruptive Changes}
\begin{description}
\item[Staff Meeting:] Students have recently started using staffs to augment their magical abilities.
\item[Parliament of Dragons:] A group of humans have arrived from overseas and are establishing new colonies.
\item[Kill the Beast:] A local farmer claims to have seen a monster lurking in the woods.
\item[Not From Around Here:] An increasing number of the guests are extraterrestrial aliens.
\end{description}

\newpage

\subsection*{Characters}
The characters are the people that the players will portray for the duration of the game.
Each player will portray one character throughout the game.
In an SWS game, the characters have gathered to discuss the ways in which they have been affected by the game's disruptive change.

\subsubsection*{Examples of Characters}
\begin{description}
\item[Staff Meeting] Professors who are preparing the curriculum for the upcoming school year.
\item[Parliament of Dragons:] Ancient wyrms who have long vied for dominance.
\item[Kill the Beast:] Villagers whose work requires them to venture into the forest.
\item[Not From Around Here:] Hospitality workers who are responsible for running the resort.
\end{description}

\newpage

\subsection*{Topics}
Every SWG features four topics that the players' characters have gathered to discuss.
Each topic addresses one aspect of the setting that will be affected by the disruptive change.
Each character has two topics of interest.
A character's final score is based on how prominently their topics of interest are in the discussion during the course of the game.

\subsubsection*{Examples of Topics}
\begin{description}
\item[Staff Meeting:] Educational Outcomes, Academic Integrity, Facilities \& Logistics, Equal Opportunity
\item[Parliament of Dragons:] Treasure, Renown, Food, Territory
\item[Kill the Beast:] Public Safety, Stewardship, Politics, Economics
\item[Not From Around Here:] Dining \& Entertainment, Guest Safety \& Comfort, Payment \& Gratuities, Reputation \& Marketing.
\end{description}

\newpage

\subsection*{Collaborative Storytelling} \label{subsection:collaborative-storytelling}
Prophecy is a storytelling game.
As such, the players' ultimate goal in the game should be to tell an interesting story.
The rules of the game are designed to help the players do so.
If at any point the players need to choose between following the rules or telling a good story, they should choose the latter.

Prophecy is also a collaborative game.
Collaboration implies shared ownership.
As such, the players should be equal partners in the playing of the game.
No player should ever try to unilaterally dictate what happens in the story and no player should ever feel like their contributions have been ignored or overruled by another player.

\subsection*{Safety Tools} \label{subsection:safety-tools}
Because stories can provoke strong emotional responses, the players should be sensitive to one another's feelings during the game.
They should avoid, or retroactively remove, any content that makes any of the players uncomfortable.

It can be difficult, however,  to identify such content. 
So, the players should establish some ground rules about what should be excluded before the game begins. 
They should also establish a way to indicate that something in the story has made someone uncomfortable and should be removed or replaced.

Players who identify problematic content do not need to explain themselves.
Because the other players may not know what the problem is, the player who identified the issue should suggest a satisfactory alternative.

\newpage

\subsection*{Playtesters} \label{subsection:playtesters}
The following people helped to refine the design of Prophecy:\vspace{-1.75ex}
\begin{multicols}{2}
\begin{itemize}
  \item Keydan Bruce
  \item Farzana Choudhury
  \item Dannielle Harden
  \item Andrew Hellyer
  \item Sarah Hewat
  \item Scott Joblin
  \item David McKenzie
  \item Paul Murray
  \item Kira Purcell
  \item Luke Purcell
  \item Jo Stephenson
  \item Brett Witty
\end{itemize}
\end{multicols}

\subsection*{Influences} \label{subsection:influences}
The following games influenced the design of Prophecy:
\begin{description}[font=\normalfont\textbullet\space, noitemsep, topsep=-1ex]
	\item[Fate:] Aspects and Objects.
	\item[Fiasco:] Scenes and scene-level resolution.
	\item[Our Last Best Hope:] Theme and narrative focus.
	\item[10 Candles:] Session modules.
	\item[Microscope:] Non-linear storytelling.
\end{description}
\vspace{1ex}
Safety Tools is based on the X-Card by John Stavropoulus.

\subsection*{Design Tools} \label{subsection:design-tools}
The following tools were used to create this booklet:
\begin{description}[font=\normalfont\textbullet\space, noitemsep, topsep=-1ex]
	\item[XeLaTeX:] Typesetting and layout.
	\item[TikZ:] Diagrams and art.
\end{description}
\vspace{1ex}
The fonts used in this booklet are TeX Gyre Heros and URW Classico.
\vspace{0.25\baselineskip}

\vfill

Contact: \href{mailto:prophecy.ttrpg@gmail.com}{prophecy.ttrpg@gmail.com}\\
\vspace{-2.5ex}\doclicenseText \hfill  \Huge{\doclicenseIcon}

\end{document}
